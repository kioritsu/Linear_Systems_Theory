\documentclass{jsarticle}

\usepackage{amssymb}
\usepackage{graphicx}
\usepackage[dvipdfmx]{color}
\usepackage{here}
\usepackage{tabularx}
\usepackage{amsmath}
\usepackage{url}
\usepackage[hang,small,bf]{caption}
\usepackage[subrefformat=parens]{subcaption}
\usepackage{tikz}
\usepackage{siunitx}
\usepackage{bm}
\usepackage{ascmac}
\usepackage[top=15truemm,bottom=20truemm,left=20truemm,right=20truemm]{geometry}
\usetikzlibrary{shapes.geometric}
\usetikzlibrary {shapes.misc}
\usetikzlibrary{positioning}
\captionsetup{compatibility=false}


\begin{document}

\rightline{2022/6/14}

\section*{7.2:ジョルダン正規形による行列のべき乗の計算}

$n\times n$ の行列 $A$ が与えられたとき、$J =PAP^{-1}$ を $A$ のジョルダン正規形とする。
\begin{equation}
  J = PAP^{-1} \Leftrightarrow A = P^{-1}JP
\end{equation}
なので、
\begin{equation}
  A^t = P^{-1}J^tP = P^{-1}
  \begin{bmatrix}
    J^t_1 & 0 & \cdots & 0\\
    0 & J^t_2 & \cdots & 0\\
    \vdots & \vdots & \ddots & \vdots\\
    0 & 0 & \cdots & j^t_l
  \end{bmatrix}P \tag{7.1}
\end{equation}
である。
ここで、$J_i$ は$A$のジョルダンブロック。
ジョルダンブロック$J_i$から$J^t_i$を計算するのは$t$の帰納法で可能であるため、簡単である。
\begin{equation}
  \begin{aligned}
  J_i =& \begin{bmatrix}
    \lambda_i&1 & 0 & \cdots & 0\\
    0 & \lambda_i &1& \cdots & 0\\
    0 &0& \lambda_i & \cdots & 0\\
    \vdots & \vdots& \vdots & \ddots & \vdots\\
    0 & 0 &0& \cdots & \lambda_i
  \end{bmatrix}_{n_i\times n_i}\\
  & \Rightarrow J^t_i = \begin{bmatrix}
    \lambda^t_i & t\lambda^{t-1}_i& \frac{t!\lambda^{t-2}_i}{(t-2)!2!}&\frac{t!\lambda^{t-3}_i}{(t-3)!3!}&\cdots&\frac{t!\lambda^{t-n_i+1}_i}{(t-n_i+1)!(n_i-1)!}\\
    0 & \lambda^t_i & t\lambda^{t-1}_i& \frac{t!\lambda^{t-2}_i}{(t-2)!2!}&\cdots&\frac{t!\lambda^{t-n_i+2}_i}{(t-n_i+2)!(n_i-2)!}\\
    0&0 & \lambda^t_i & t\lambda^{t-1}_i&\cdots&\frac{t!\lambda^{t-n_i+3}_i}{(t-n_i+3)!(n_i-3)!}\\
    \vdots&\vdots&\vdots&\ddots&\ddots &\vdots\\
    0&0&0&0&\ddots&t\lambda^{t-1}_i\\
    0&0&0&0&\ddots&t\lambda^{t}_i\\
  \end{bmatrix}
\end{aligned}\tag{7.2}
\end{equation}
上の式は $O! = 1 $と$J^t_i$のエントリのうち負の数の階乗を含むものを0と仮定している。

式(7.2) の $J^t_i$ のエントリを調べると,次のような結論が得られる。
\begin{enumerate}
  \item $\lambda_i$のマグニチュードが1より厳密に小さいとき、$t\rightarrow \infty$として$J^t_i\rightarrow 0$
  \item $\lambda_i$のマグニチュードが1に等しく、
  ジョルダンブロックが$1\times 1$(すなわち$n_i = 1$)
  のとき、$t\rightarrow \infty$として$J^t_i$は有界である。
  \item $\lambda_i$のマグニチュードが1より厳密に大きいか、
  または$\lambda_i$のマグニチュードが1に等しく、
  ジョルダンブロックが$1\times 1$より大きいとき、
  $t\rightarrow \infty$として$J^t_i$は非有界である。

\end{enumerate}

これらの考察は式(7.1) と共に、
$A$の固有値と$t\rightarrow \infty$においての$A^t$に起こること
の関係について、
以前に見たことを確認し、さらなる考察を与えるもの。

\begin{enumerate}
  \item $A$のすべての固有値のマグニチュードが 1 よりも厳密に小さいとき、
  $t\rightarrow \infty$としてすべての$J^t_i\rightarrow 0$であり、
  したがって$t\rightarrow \infty$として$A \rightarrow 0 $となる。
  \item $A$のすべての固有値のマグニチュードが 1 より小さいか等しく、
  マグニチュードが 1 に等しい固有値に対応する
  すべてのジョルダンブロックが $1 \times 1$ であるとき、
  $t\rightarrow \infty$として全ての$J^t_i$は有界であり、
  その結果、$t\rightarrow \infty$として$A^t$も有界。
  \item $A$の固有値のうち少なくとも1つが1より大きい、
  または1に等しい大きさで、対応するジョルダンブロックが
  $1 \times 1$ より大きいとき、$t\rightarrow \infty$として$A^t$は非有界。


\end{enumerate}


\end{document}