\documentclass{jsarticle}

\usepackage{amssymb}
\usepackage{graphicx}
\usepackage[dvipdfmx]{color}
\usepackage{here}
\usepackage{tabularx}
\usepackage{amsmath}
\usepackage{url}
\usepackage[hang,small,bf]{caption}
\usepackage[subrefformat=parens]{subcaption}
\usepackage{tikz}
\usepackage{siunitx}
\usepackage{bm}
\usepackage{ascmac}
\usepackage[top=15truemm,bottom=20truemm,left=20truemm,right=20truemm]{geometry}
\usetikzlibrary{shapes.geometric}
\usetikzlibrary {shapes.misc}
\usetikzlibrary{positioning}
\captionsetup{compatibility=false}


\begin{document}

\rightline{2022/6/15}

\section{EX7.4:行列のべき乗と指数} 
行列$A$を考える。
\begin{equation}
  A = 
  \begin{bmatrix}
    2&2&0\\
    0&0&0\\
    0&0&2
  \end{bmatrix}
\end{equation}
$A^{100}$と$e^{t A}$を計算せよ。
ヒント:$A$を対角化。
\;\\
$A$を対角化する。
\begin{equation}
  \text{det}(A - \lambda) = \text{det}
  \begin{bmatrix}
    2-\lambda&2&0\\
    0&-\lambda&0\\
    0&0&2-\lambda
  \end{bmatrix}
   = -\lambda(2-\lambda)^2
\end{equation}
$\lambda = 2$を考える。
\begin{equation}
  A-\lambda = \begin{bmatrix}
    0&2&0\\
    0&-2&0\\
    0&0&0
  \end{bmatrix}
\end{equation}
であるため、固有ベクトルは
$\begin{bmatrix}
  1\\0\\0
\end{bmatrix},
\begin{bmatrix}
  0\\0\\1
\end{bmatrix}$\\
$\lambda = 0$を考える。
\begin{equation}
  A-\lambda = \begin{bmatrix}
    2&2&0\\
    0&0&0\\
    0&0&2
  \end{bmatrix}
\end{equation}
であるため、固有ベクトルは
$\begin{bmatrix}
  -1\\1\\0
\end{bmatrix},$
よって、
\begin{equation}
  P = \begin{bmatrix}
    1&0&-1\\
    0&0&1\\
    0&1&0
  \end{bmatrix},
  P^{-1} = \begin{bmatrix}
    1&1&0\\
    0&0&1\\
    0&1&0
  \end{bmatrix}
\end{equation}
\begin{equation}
  D = P^{-1}AP = \begin{bmatrix}
    2&0&0\\
    0&2&0\\
    0&0&0
  \end{bmatrix}
\end{equation}
それぞれ計算する。
\begin{equation}
  A^{100} = PD^{100}P^{-1} = 
  \begin{bmatrix}
    1&0&-1\\
    0&0&1\\
    0&1&0
  \end{bmatrix}
  \begin{bmatrix}
    2^{100}&0&0\\
    0&2^{100}&0\\
    0&0&0
  \end{bmatrix}
  \begin{bmatrix}
    1&1&0\\
    0&0&1\\
    0&1&0
  \end{bmatrix}
  =\begin{bmatrix}
    2^{100}&2^{100}&0\\
    0&0&0\\
    0&0&2^{100}
  \end{bmatrix}
\end{equation}
\begin{equation}
  e^{tA} = Pe^{tD}P^{-1} = 
  \begin{bmatrix}
    1&0&-1\\
    0&0&1\\
    0&1&0
  \end{bmatrix}
  \begin{bmatrix}
    e^{2t}&0&0\\
    0&e^{2t}&0\\
    0&0&0
  \end{bmatrix}
  \begin{bmatrix}
    1&1&0\\
    0&0&1\\
    0&1&0
  \end{bmatrix}
   = \begin{bmatrix}
    e^{2t}&e^{2t}&0\\
    0&0&0\\
    0&0&e^{2t}
  \end{bmatrix}
\end{equation}










\end{document}