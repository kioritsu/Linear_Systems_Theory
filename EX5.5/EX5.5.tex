\documentclass{jsarticle}

\usepackage{amssymb}
\usepackage{graphicx}
\usepackage[dvipdfmx]{color}
\usepackage{here}
\usepackage{tabularx}
\usepackage{amsmath}
\usepackage{url}
\usepackage[hang,small,bf]{caption}
\usepackage[subrefformat=parens]{subcaption}
\usepackage{tikz}
\usepackage{siunitx}
\usepackage{bm}
\usepackage{ascmac}
\usepackage[top=15truemm,bottom=20truemm,left=20truemm,right=20truemm]{geometry}
\usetikzlibrary{shapes.geometric}
\usetikzlibrary {shapes.misc}
\usetikzlibrary{positioning}
\captionsetup{compatibility=false}


\begin{document}

\rightline{2022/6/7}

\section*{EX5.5:LTV システムの解} 

状態遷移行列$\Phi(t,\tau)$を持つ、斉次線形時変システム
\begin{equation}
  \dot{x} =A(t)x\;,\;x(0) = x_0
\end{equation}
を考える。また、入力 $x ( t )$ が斉次システムの状態である非斉次システム
\begin{equation}
  \dot{z} = A(t)z + x(t) \;,\; z(0) = z_0
\end{equation}
も考える。
\begin{itemize}
  \item[(a)] $x_0 , z_0,\Phi$ の関数として $x ( t )$ と $z ( t ) $を計算せよ。答えに積分は含まれない。
  \item[(b)] 与えられた時間 $T > 0 $において、$z ( T ) = 0$ となるような $x_0$ と $z_0$ はどのように関係づけられるか?
\end{itemize}
\;\\
(a)\;\;定理5.1より、
\begin{equation}
  x(t) = \Phi(t,0)x_0
\end{equation}
定理5.2より、
\begin{equation}
  \begin{aligned}
    z(t) = &\Phi(t,0)z_0 + \int^t_0\Phi(t,\tau)x(\tau)d\tau\\
     = & \Phi(t,0)z_0 + \int^t_0\Phi(t,\tau)\Phi(\tau,0)x_0 d\tau\\
     = &  \Phi(t,0)z_0 + tx_0\Phi(t,0)
  \end{aligned} 
\end{equation}

(b)\;\;式(4)より、$z ( T ) = 0$は、
\begin{equation}
  \begin{aligned}
    z(T) = \Phi(T,0)z_0 + tx_0\Phi(T,0) = 0\\
    z_0 + tx_0 = 0
  \end{aligned}
\end{equation}


\end{document}