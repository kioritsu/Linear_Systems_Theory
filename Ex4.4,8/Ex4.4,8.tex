\documentclass{jsarticle}

\usepackage{amssymb}
\usepackage{graphicx}
\usepackage[dvipdfmx]{color}
\usepackage{here}
\usepackage{tabularx}
\usepackage{amsmath}
\usepackage{url}
\usepackage[hang,small,bf]{caption}
\usepackage[subrefformat=parens]{subcaption}
\usepackage{tikz}
\usepackage{siunitx}
\usepackage{bm}
\usepackage{ascmac}
\usepackage[top=15truemm,bottom=20truemm,left=20truemm,right=20truemm]{geometry}
\usetikzlibrary{shapes.geometric}
\usetikzlibrary {shapes.misc}
\usetikzlibrary{positioning}
\captionsetup{compatibility=false}


\begin{document}

\rightline{2022/6/1}
\section*{Ex4.4(LTIシステムの出力のZ変換)}
\begin{equation}
  x^{+}=A x+B u, \quad y=C x+D u \tag{DLTI}
\end{equation}
(DLTI)への出力のZ変換が
\begin{equation}
  \begin{aligned}
    \hat{y}(z)=\hat{\Psi}(z) x(0)+\hat{G}(z) \hat{u}(z), \quad& \hat{\Psi}(z):=C(z I-A)^{-1}z\\
    & \hat{G}(z):=C(z I-A)^{-1} B+D 
  \end{aligned}
\end{equation}
で与えられることを示せ。
\;\\
p41のNote7より、$Z\left[x(t+1)\right] = z(\hat{x}(z)-x(0))$。また、Z変換の線形性より、
(DLTI)のZ変換はそれぞれ、
\begin{equation}
  z(\hat{x}(z)-x(0)) = A\hat{x}(z)+B\hat{u}(z)\;,\;\;
  \hat{y}(z) = C\hat{x}(z)+D\hat{u}(z)
\end{equation}
で得られる。
\begin{equation}
  \hat{x}(z) = {(zI-A)}^{-1}B\hat{u}(z)+{(zI-A)}^{-1}zx(0)
\end{equation}
式変形して得られた$\hat{x}(z)$を代入して、
\begin{equation}
  \hat{y}(z) = \{C(z I-A)^{-1}z\}\hat{u}(z) + \{C(z I-A)^{-1} B+D\}\hat{u}(z)
\end{equation}


\newpage

\section*{Ex4.8(同等の実現)}
次の2つのシステムを考える。
\begin{equation}
  \dot{x} = 
  \begin{bmatrix}
    2&1&2\\
    0&2&2\\
    0&0&1
  \end{bmatrix}x+
    \begin{bmatrix}
      1\\1\\0
    \end{bmatrix}u\;,\;\;y=
    \begin{bmatrix}
      1&-1&0
    \end{bmatrix}x
\end{equation}
\begin{equation}
  \dot{x} = 
  \begin{bmatrix}
    2&1&1\\
    0&2&1\\
    0&0&-1
  \end{bmatrix}x+
    \begin{bmatrix}
      1\\1\\0
    \end{bmatrix}u\;,\;\;y=
    \begin{bmatrix}
      1&-1&0
    \end{bmatrix}x
\end{equation}
\begin{itemize}
  \item[(a)] これらのシステムはゼロ状態等価であるか。
  \item[(b)] それらは代数的に等価であるか。
\end{itemize}
\;\\

(a) それぞれの伝達関数を求める。
\begin{equation}
  \left\{\begin{array} { l } 
  { \dot { x } = A x + B u  } \\
  { y = C x + D u }
  \end{array} 
  \right.
  \end{equation}
\begin{equation}
  \hat{G}(s) = C(sI-A)^{-1}B+D\tag{4.3}
\end{equation}
式(5)\;\;
\begin{equation}
  (sI-A) = \begin{bmatrix}
    s-2& -1& -2\\
    0  &s-2& -2\\
    0  & 0 &s-1
  \end{bmatrix}
\end{equation}
\begin{equation}
  \text{det}(sI-A) = (s-2)^2(s-1)
\;,\;
  \text{cof}(sI-A) = \begin{bmatrix}
    (s-2)(s-1)& 0 & 0\\
    (s-1)&(s-2)(s-1)& 0\\
    2(s-1) &2(s-2)&(s-2)^2
  \end{bmatrix}
\end{equation}
\begin{equation}
  {(sI-A)}^{-1} = \frac{1}{(s-2)^2(s-1)}
  \begin{bmatrix}
    (s-2)(s-1)& (s-1) & 2(s-1)\\
    0&(s-2)(s-1)& 2(s-2)\\
    0 &0&(s-2)^2
  \end{bmatrix}
\end{equation}
\begin{equation}
  \hat{G}_5(s) = 
  \begin{bmatrix}
    1&-1&0
  \end{bmatrix}
  \frac{1}{(s-2)^2(s-1)}
  \begin{bmatrix}
    (s-2)(s-1)& (s-1) & 2(s-1)\\
    0&(s-2)(s-1)& 2(s-2)\\
    0 &0&(s-2)^2
  \end{bmatrix}
  \begin{bmatrix}
    1\\1\\0
  \end{bmatrix}
  = \frac{1}{(s-2)^2}
\end{equation}

式(6)\;\;
\begin{equation}
  (sI-A) = \begin{bmatrix}
    s-2& -1& -1\\
    0  &s-2& -1\\
    0  & 0 &s-1
  \end{bmatrix}
\end{equation}
\begin{equation}
  \text{det}(sI-A) = (s-2)^2(s-1) \;,\;
  \text{cof}(sI-A) = \begin{bmatrix}
    (s-2)(s-1)& 0 & 0\\
    s-1&(s-2)(s-1)& 0\\
    s-1 &s-2&(s-2)^2
  \end{bmatrix} 
\end{equation}
\begin{equation}
  {(sI-A)}^{-1} = \frac{1}{(s-2)^2(s-1)}
  \begin{bmatrix}
    (s-2)(s-1)& s-1 & s-1\\
    0&(s-2)(s-1)&s-2\\
    0 &0&(s-2)^2
  \end{bmatrix}
\end{equation}



\begin{equation}
  \hat{G}_6(s) = 
  \begin{bmatrix}
    1&-1&0
  \end{bmatrix}
  \frac{1}{(s-2)^2(s-1)}
  \begin{bmatrix}
    (s-2)(s-1)& s-1 & s-1\\
    0&(s-2)(s-1)&s-2\\
    0 &0&(s-2)^2
  \end{bmatrix}
  \begin{bmatrix}
    1\\1\\0
  \end{bmatrix}
  = \frac{1}{(s-2)^2}
\end{equation}
よって、伝達関数が等しいため、ゼロ状態等価である。\\
\newpage

(b)\;\;それらは代数的に等価であるかを考える。
\begin{equation}
  \left\{\begin{array} { l } 
  { \dot { x } = A x + B u } \\
  { y = C x + D u }
  \end{array} 
  \right.
  \;\;
  \left\{\begin{array} { l } 
    { \dot { x } = \bar{A} x + \bar{B} u } \\
    { y = \bar{C} x + \bar{D} u }
    \end{array} 
    \right.
\end{equation}
2つのLTIシステムについて、以下を満たす非特異行列$T$が存在する時、代数的等価であるという。
\begin{equation}
  \bar{A} = TAT^{-1}\;,\;\bar{B} = TB \;,\; \bar{C} = CT^{-1}\;,\;\bar{D} = D
\end{equation}
今回の問題では、
\begin{equation}
  \begin{bmatrix}
    2&1&1\\
    0&2&1\\
    0&0&-1
  \end{bmatrix} = T\begin{bmatrix}
    2&1&2\\
    0&2&2\\
    0&0&1
  \end{bmatrix}T^{-1}\;,\;\begin{bmatrix}
    1\\1\\0
  \end{bmatrix} = T\begin{bmatrix}
    1\\1\\0
  \end{bmatrix} \;,\; \begin{bmatrix}
    1&-1&0
  \end{bmatrix}= \begin{bmatrix}
    1&-1&0
  \end{bmatrix}T^{-1}
\end{equation}
を満たす$T$が存在するかを考える。
式(5),(6)の行列$A$部分の全ての要素が等しくないため、代数的に等価ではない。


\end{document}