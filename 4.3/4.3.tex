\documentclass{jsarticle}

\usepackage{amssymb}
\usepackage{graphicx}
\usepackage[dvipdfmx]{color}
\usepackage{here}
\usepackage{tabularx}
\usepackage{amsmath}
\usepackage{url}
\usepackage[hang,small,bf]{caption}
\usepackage[subrefformat=parens]{subcaption}
\usepackage{tikz}
\usepackage{siunitx}
\usepackage{bm}
\usepackage{ascmac}
\usepackage[top=15truemm,bottom=20truemm,left=20truemm,right=20truemm]{geometry}
\usetikzlibrary{shapes.geometric}
\usetikzlibrary {shapes.misc}
\usetikzlibrary{positioning}
\captionsetup{compatibility=false}


\begin{document}

\rightline{2022/5/31}

\section*{4.3 : 初等実現論} 
\begin{itembox}[l]{定義4.1:(Realization)}
  伝達関数 $\hat{G} ( s )$ が与えられたとき、連続時間または離散時間LTI状態空間システム 
  \begin{equation}
    \left\{\begin{array} { l } 
    { \dot { x } = A x + B u , } \\
    { y = C x + D u }
    \end{array} \quad \text { or } \quad \left\{\begin{array}{l}
    x^{+}=A x+B u \\
    y=C x+D u
    \end{array}\right.\right. \tag{LTI}
    \end{equation}
  は、それぞれ 式(4.3)が成り立つとき $\hat{G} ( s )$の実現(realization)であると呼ぶ。
  \begin{equation}
    \hat{G}(s) = C(sI-A)^{-1}B+D\tag{4.3}
  \end{equation}
  離散時間系では、式(4.3) の $s$ を $z$ で置き換えることになる。
\end{itembox}
一般に,多くのシステムが同じ伝達関数を実現する可能性があり、それにより次の定義が得られる。

\begin{itembox}[l]{定義4.2:(ゼロ状態等価)}
  2つの状態空間系が同じ伝達関数を実現しているとき、ゼロ状態等価(zero-state equivalence)であると呼ぶ。
   これは、すべての入力に対して同じ強制応答を示すことを意味する。
\end{itembox}

\section*{4.3.1:実現から伝達関数}
LTI状態空間システムで実現できる伝達関数にはどのような種類があるのかを考える。
(LTI)によって実現される伝達関数を計算することを試みる。
\begin{equation}
  M^{-1} = \frac{1}{\text{det}M}\text{adj}M,\;\;\;
  \text{adj}M:={(\text{cof}M)}',\;\;\;
  \text{cof}M:= [\text{cof}_{ij}M]
\end{equation}
ここで$\text{cof} M$は余因子行列を表し、その$ij$番目のエントリは$M$の$ij$番目の余因子、
すなわち$i$行と$j$列を削除して得られたMの部分行列の行列式に$(-1)^{i+j}$を掛けたもの。
\begin{itemize}
  \item $\text{adj}M$:adjoint matrix,随伴行列:cofactor matrixの転置行列
  \item $\text{cof}M$:cofactor matrix,余因子行列 :(i,j)成分が (i,j)余因子(cofactor)である行列
\end{itemize}

したがって、式(4.3)は
\begin{equation}
  C(sI-A)^{-1}B+D = \frac{1}{\text{det}(sI-A)}C\text{adj}(sI-A)B+D
\end{equation}
分母$\text{det}(sI-A)$は$A$の特性多項式と呼ばれる$n$次多項式で、
その根はAの固有値である。
随伴行列$\text{adj}(sI-A)$は $( n - 1 ) \times ( n - 1 )$ 行列の行列式を含むので、
その要素は $n - 1$ 程度以下の多項式となる。したがって、$C\text{adj}(sI-A)B$ の要素は、
$n - 1$ 以下の次数の多項式の線形結合になる。
したがって、
\begin{equation}
  C\text{adj}(sI-A)B
\end{equation}
の要素はすべて、分母の次数が分子の次数より厳密に大きい多項式の比となる。
このような形式の関数を厳密真有理関数(strictly proper rational function)と呼ぶ。

$D \neq 0$のとき、464
\begin{equation}
  \frac{1}{\text{det}(sI-A)}C\text{adj}(sI-A)B+D
\end{equation}
の分子の多項式は分母と同じ次数のものがある
(ただしそれ以上ではない)。これを真有理関数(proper rational function)と呼ぶ。
したがって、LTI状態空間系は真有理関数のみを実現することができると結論づけられる。
実際には、LTI状態空間系は全ての真有理関数を実現できることがわかります。
\begin{itembox}[l]{定理4.3:( MIMO 実現 )}
  伝達関数 $\hat{G} ( s )$は、$\hat{G} ( s )$が有理関数である場合にのみ、
  LTI 状態空間系によって実現することができる。
\end{itembox}

\section*{4.3.2:伝達関数から実現}
$\hat{G} ( s )$がLTIシステムによって実現されるためには、
真有理関数であることが必要であることが前の章で分かった。
その逆を証明するには、任意に与えられた真有理関数$\hat{G} ( s )$を
実現するLTIシステムの構築方法を示す必要がある。

次のような手順で行う。
\begin{enumerate}
  \item $m\times k$ 行列$\hat{G} ( s )$を式(4.4)のように分解し、
  \begin{equation}
    \hat{G} ( s ) = \hat{G}_{sp}( s )+D \tag{4.4}
  \end{equation}
  $\hat{G} ( s )$は厳密真有理関数、$D$は式(4.5)のように、定数行列とする。
  \begin{equation}
    D := \lim_{s \rightarrow \infty}{\hat{G} ( s )} \tag{4.5}
  \end{equation}
  行列$D$は状態空間系の一部であり、下を満たすように$A,B,C$を選ぶこととする。
  \begin{equation}
    \hat{G}_{sp}( s ) = C(sI-A)^{-1}B
  \end{equation}

  \item  $\hat{G}_{sp}( s )$のすべての要素のモニック最小公分母$d(s)$を求める。
  \begin{equation}
    d(s)=s^n + \alpha_1s^{n-1} + \alpha_2 s^{n-2}+\cdots+\alpha_{n-1}s+\alpha_n
  \end{equation}
  
  \item $\hat{G}_{sp}( s )$を式(4.6)のように展開する。
  ここで、$N_i$は定数$m \times k$行列。
  \begin{equation}
    \hat{G}_{sp}( s ) = \frac{1}{d(s)}\left[ N_1s^{n-1} + N_2s^{n-2} + \cdots + N_{n-1}s + N_n \right] \tag{4.6}
  \end{equation}

  \item 式(4.7a),(4.7b)を選ぶ。
  \begin{equation}
    \begin{array}{l}
    A=\left[\begin{array}{ccccc}
    -\alpha_{1} I_{k \times k} & -\alpha_{2} I_{k \times k} & \cdots & -\alpha_{n-1} I_{k \times k} & -\alpha_{n} I_{k \times k} \\
    I_{k \times k} & 0_{k \times k} & \cdots & 0_{k \times k} & 0_{k \times k} \\
    0_{k \times k} & I_{k \times k} & \cdots & 0_{k \times k} & 0_{k \times k} \\
    \vdots & \vdots & \ddots & \vdots & \vdots \\
    0_{k \times k} & 0_{k \times k} & \cdots & I_{k \times k} & 0_{k \times k}
    \end{array}\right]_{n k \times n k}\\
  \end{array}\tag{4.7a}
  \end{equation}
    \begin{equation}
    B=\left[\begin{array}{c}
    I_{k \times k} \\
    0_{k \times k} \\
    \vdots \\
    0_{k \times k} \\
    0_{k \times k}
    \end{array}\right]_{n k \times k}, \quad C=\left[\begin{array}{lllll}
    N_{1} & N_{2} & \cdots & N_{n-1} & N_{n}
    \end{array}\right]_{m \times n k}\tag{4.7b}
  \end{equation}

\end{enumerate}
これを制御可能な標準形(controllable canonical form)の実現と呼ぶ。
\newpage
\begin{itembox}[l]{proposition4.1}
  式(4.5)、(4.7) で定義される行列 $( A , B , C , D )$ は$\hat{G}( s )$の実現である。
\end{itembox}
証明\;\;
まず、式(4.8)の解であるベクトル
$Z(s) = \left[ Z'_1\;Z'_2\;\cdots\;Z'_n \right]' := (sI-A)^{-1}B$
を計算する。
\begin{equation}
  (s I-A) Z(s)=B \Leftrightarrow\left\{\begin{array}{l}
  \left(s+\alpha_{1}\right) Z_{1}+\alpha_{2} Z_{2}+\cdots+\alpha_{n-1} Z_{n-1}+\alpha_{n} Z_{n}=I_{k \times k} \\
  s Z_{2}-Z_{1}=0, s Z_{3}-Z_{2}=0, \ldots, s Z_{n}=Z_{n-1}
  \end{array}\right.\tag{4.8}
\end{equation}
式(4.8)の右辺の下の行より、
\begin{equation}
  Z_k = \frac{1}{s^{k-1}}Z_1
\end{equation}
これを式(4.8)の右辺の上の行に代入すると
\begin{equation}
  \left(s+\alpha_{1}+\frac{\alpha_{2}}{s}+\cdots+\frac{\alpha_{n-1}}{s^{n-2}}+\frac{\alpha_{n}}{s^{n-1}}\right) Z_{1}=I_{k \times k}
\end{equation}
左辺の多項式は$\frac{d(s)}{s^{n-1}}$で与えられるので、
\begin{equation}
  Z(s)=\left[\begin{array}{c}
  Z_{1} \\
  Z_{2} \\
  \vdots \\
  Z_{n}
  \end{array}\right]=\frac{1}{d(s)}\left[\begin{array}{c}
  s^{n-1} I_{k \times k} \\
  s^{n-2} I_{k \times k} \\
  \vdots \\
  I_{k \times k}
  \end{array}\right]
\end{equation}
となる。 
これと式(4.6)より、
\begin{equation}
  C(s I-A)^{-1} B=C Z(s)=\frac{1}{d(s)}\left[\begin{array}{lllll}
  N_{1} & N_{2} & \cdots & N_{n-1} & N_{n}
  \end{array}\right]\left[\begin{array}{c}
  s^{n-1} I_{k \times k} \\
  s^{n-2} I_{k \times k} \\
  \vdots \\
  I_{k \times k}
  \end{array}\right]=\hat{G}_{s p}(s)
\end{equation}
となる。最後に式(4.4)を用いて、$\hat{G}(s) = C(sI-A)^{-1}B+D$が分かる。$\Box$

\section*{4.3.3:SISOの場合} 
SISOの厳密適切系では、4.3.2節で示した構成が非常に簡単になり、
検査によって実現が決定できる。次の定理(演習4.6で証明)は、この観測を要約したもの。

\begin{itembox}[l]{定理 4.4:SISO 実現}
  厳密に適切なSISO伝達関数
  \begin{equation}
    \hat{g}(s)=\frac{\beta_{1} s^{n-1}+\beta_{2} s^{n-2}+\cdots+\beta_{n-1} s+\beta_{n}}{s^{n}+\alpha_{1} s^{n-1}+\alpha_{2} s^{n-2}+\cdots+\alpha_{n-1} s+\alpha_{n}}
  \end{equation}
  は、
  \begin{equation}
    A=\left[\begin{array}{ccccc}
      -\alpha_{1} & -\alpha_{2} & \cdots & -\alpha_{n-1} & -\alpha_{n} \\
      1 & 0 & \cdots & 0 & 0 \\
      0 & 1 & \cdots & 0 & 0 \\
      \vdots & \vdots & \ddots & \vdots & \vdots \\
      0 & 0 & \cdots & 1 & 0
      \end{array}\right]_{n \times n}, \quad B=\left[\begin{array}{c}
      1 \\
      0 \\
      \vdots \\
      0 \\
      0
      \end{array}\right]_{n \times 1}
      $$
      $$
      C=\left[\begin{array}{lllll}
      \beta_{1} & \beta_{2} & \cdots & \beta_{n-1} & \beta_{n}
      \end{array}\right]_{1 \times n}
  \end{equation}
  または
  \begin{equation}
    \bar{A}=\left[\begin{array}{ccccc}
      -\alpha_{1} & 1 & 0 & \cdots & 0 \\
      -\alpha_{2} & 0 & 1 & \cdots & 0 \\
      -\alpha_{3} & 0 & 0 & \cdots & 0 \\
      \vdots & \vdots & \vdots & \ddots & \vdots \\
      -\alpha_{n} & 0 & 0 & \cdots & 0
      \end{array}\right]_{n \times n}, \quad \bar{B}=\left[\begin{array}{c}
      \beta_{1} \\
      \beta_{2} \\
      \beta_{3} \\
      \vdots \\
      \beta_{n}
      \end{array}\right]_{n \times 1}, \bar{C}=\left[\begin{array}{lllll}
      1 & 0 & \ldots & 0 & 0
      \end{array}\right]_{1 \times n}
  \end{equation}
  のいずれかの実現を認める。
\end{itembox}

\end{document}