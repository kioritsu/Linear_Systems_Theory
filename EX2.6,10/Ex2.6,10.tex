\documentclass{jsarticle}
\usepackage{amssymb}
\usepackage{graphicx}
\usepackage[dvipdfmx]{color}
\usepackage{here}
\usepackage{tabularx}
\usepackage{amsmath}
\usepackage{mathtools}
\usepackage{url}
\usepackage[hang,small,bf]{caption}
\usepackage[subrefformat=parens]{subcaption}
\usepackage{tikz}
\usepackage{siunitx}
\usepackage{bm}
\usepackage{ascmac}
\usepackage[top=15truemm,bottom=20truemm,left=20truemm,right=20truemm]{geometry}
\usetikzlibrary{shapes.geometric}
\usetikzlibrary {shapes.misc}
\usetikzlibrary{positioning}
\captionsetup{compatibility=false}

\begin{document}

\rightline{2022/5/25}

\section*{Ex2.6 (平衡付近の局所的な線形化:one-link robot)} 

図2.9のようなワンリンクロボットを考える。
ここで、$\theta$はリンクの水平に対する角度、$\tau$は基礎にかかるトルク、$(x,y)$は先端の位置、$l$はリンクの長さ、
$I$は慣性モーメント、$m$は先端の質量、$g$は重力加速度、$b$は摩擦係数を表します。この系は以下の式に従って進化する。
\begin{equation}
  I\ddot{\theta} = -b\dot{\theta} - gm\cos{\theta} + \tau
\end{equation}

(a) $u = \tau$ を入力とし、先端の垂直位置 $y$ を出力としたときの系の状態空間モデルを計算しなさい。
先端部の水平位置 $x$ と混同しないように状態ベクトルを $\bm{z}$ とし、適当な関数 $f, g$ に対して
\begin{equation}
  \dot{z} = f(z,u),y = g(z,u)
\end{equation}
という形でモデルを記述せよ。
【ヒント:出力の式を忘れないように。】
\;\\
\;\\
まず、状態ベクトルを
$\bm{z} = \left[ \begin{array}{c} z_1\\z_2 \end{array} \right] = \left[ \begin{array}{c} \theta\\ \dot{\theta} \end{array}\right]$とする。
ここで、$\dot{\bm{z}},y$を考えると、
\;\\
\;\\
\begin{equation}
  \dot{\bm{z}} = \left[
    \begin{array}{c}
      \dot{\theta}\\
      -\frac{b}{I}\dot{\theta}-\frac{gm}{I}\cos{\theta}+\frac{\tau}{I}
    \end{array}
  \right]
  = \left[
    \begin{array}{c}
      z_2\\
      -\frac{b}{I}z_2-\frac{gm}{I}\cos{z_1}+\frac{u}{I}
    \end{array}
  \right]
  = f(\bm{z},u)
\end{equation}

\begin{equation}
  y = l\sin{\theta} = l\sin{z_1} = g(\bm{z},u)
\end{equation}



(b) $\theta(t) = \frac{\pi}{2},\tau(t) = 0$が系の解であることを示し、この解を中心とした線形化を計算しなさい。
この解答から、$y$ からのフィードバックだけで、先端位置をこの形に近づけようとした場合、問題があるかどうか予測できますか?
\;\\

$\theta(t) = \frac{\pi}{2},\tau(t) = 0$は式(1)を満たすため、系の解である。\par
定義2.5より、この解を中心とした線形化を計算する。
\begin{equation}
  \dot{\partial \bm{z}} = A(t)\partial\bm{z} +B(t)\partial u\;,\;
  \partial y =C(t) \partial \bm{z} + D(t)\partial u
\end{equation}

\begin{equation*}
  \begin{array}{cc}
    \frac{\partial f(\bm{z},u)}{\partial\bm{z}} = 
    \left[
    \begin{array}{ccc}
      0 & 1 \\
      \frac{gm}{I}\sin{z_1} & -\frac{b}{I} 
    \end{array}
    \right] 
    \;\;,&\;\;
    \frac{\partial f(\bm{z},u)}{\partial u} = 
    \left[
    \begin{array}{c}
      0\\
      \frac{1}{I}
    \end{array}
    \right]\\
    \;\\
    \frac{\partial g(\bm{z},u)}{\partial \bm{z}} = 
    \left[
    \begin{array}{c}
      l\cos{z_1}\\
      0
    \end{array}
    \right]
    \;\;,&\;\;
    \frac{\partial g(\bm{z},u)}{\partial u} = 0\\
  \end{array}
\end{equation*}

$\bm{z},u$に解を代入。

\begin{equation}
  \begin{array}{cc}
    A(t) = 
    \left[
    \begin{array}{ccc}
      0 & 1 \\
      \frac{gm}{I} & -\frac{b}{I} 
    \end{array}
    \right]
    \;\;,&\;\;
    B(t) = 
    \left[
    \begin{array}{c}
      0\\
      \frac{1}{I}
    \end{array}
    \right]\\
    \;\\

    C(t) = \bm{0}
    \;\;,&\;\;
    D(t) = 0\\
  \end{array} 
\end{equation}

\begin{equation}
  \dot{\partial \bm{z}} = 
  \left[
    \begin{array}{ccc}
      0 & 1 \\
      \frac{gm}{I} & -\frac{b}{I} 
    \end{array}
    \right]
    \partial \bm{z} + 
    \left[
    \begin{array}{c}
      0\\
      \frac{1}{I}
    \end{array}
    \right]
    \partial u 
    \;\;,\;\;
    \partial y = 0
\end{equation}

出力$y$の変化がないため、$y$からのフィードバックだけでは問題がありそう。

\newpage

\section*{Ex2.10(厳密フィードバック形式のシステムに対するフィードバック線形化)}

2.4.4節で説明した、
次数2の厳密なフィードバック形式のシステムに対する手順を、 
\begin{equation}
  \begin{array}{c}
    \dot{x}_1 = f_1(x_1) + x_2\\
    \dot{x}_2 = f_2(x_1,x_2) +x_3\\
    \vdots\\
    \dot{x}_n = f_n(x_1,x_2,\dots,x_n)+u
  \end{array}
\end{equation}
で与えられる任意の次数$n \geq 2$の厳密なフィードバックのシステムへ拡張せよ。
\;\\

変数$z_2$を下のように定義する。
\begin{equation}
  z_2 := \dot{x}_1 = f_1(x_1) + x_2
\end{equation}
微分を考える。
\begin{equation}
  \begin{aligned}
    \dot{z}_2 &= \frac{\partial f_1}{\partial x_1}(x_1)\dot{x}_1+\dot{x}_2\\
    &= \frac{\partial f_1}{\partial x_1}(x_1)( f_1(x_1) + x_2)+(f_2(x_1,x_2) +x_3)\\
  \end{aligned}
\end{equation}
変数$z_3$を下のように定義する。
\begin{equation}
  z_3 := \dot{z}_2 = \frac{\partial f_1}{\partial x_1}(x_1)( f_1(x_1) + x_2)+(f_2(x_1,x_2) +x_3)
\end{equation}
微分を考える。
\begin{equation}
  \begin{aligned}
    \dot{z}_3 = & \frac{\partial^2 f_1}{\partial x_1^2}(x_1)\dot{x}_1( f_1(x_1) + x_2) 
    + \frac{\partial f_1}{\partial x_1} (x_1)  \{ \frac{\partial f_1}{\partial x_1} (x_1)\dot{x}_1 + \dot{x}_2 \} 
    + \frac{\partial f_2} {\partial x_1} (x_1,x_2)\dot{x}_1 + \frac{\partial f_2} {\partial x_2} (x_1,x_2)\dot{x}_2 + \dot{x}_3\\
    = & \frac{\partial^2 f_1}{\partial x_1^2}(x_1)( f_1(x_1) + x_2)^2
    +  \{ \frac{\partial f_1}{\partial x_1} (x_1) \}^2( f_1(x_1) + x_2)
    + \frac{\partial f_1}{\partial x_1} (x_1) (f_2(x_1,x_2) +x_3)\\
    &+ \frac{\partial f_2} {\partial x_1} (x_1,x_2)( f_1(x_1) + x_2) + \frac{\partial f_2} {\partial x_2} (x_1,x_2) (f_2(x_1,x_2) +x_3)
    +(f_2(x_1,x_2,x_3) +x_4)
  \end{aligned}
\end{equation}

同様に、$z_i = \dot{z}_{i-1}$と考えると、$z_{n}$の微分は
\begin{equation}
  \dot{z}_{n} =  \frac{\partial^{n-1}f_1}{\partial x_1^2}(x_1)( f_1(x_1) + x_2)^{n-1} +\dots + f_n(x_1,x_2,\dots,x_n)+u
\end{equation}
となり、$\dot{z}_n = v$となるように$v$を定めると、
\begin{equation}
  \begin{array}{c}
    \dot{x}_1 = z_2\\
    \dot{z}_2 = z_3\\
    \vdots\\
    \dot{z}_n = v
  \end{array}
\end{equation}
となる。
\;\\
\;\\
$\dot{x}_{i-1} = z_{i}$とした場合、$z_{i} = v_{i}$をみたすような$v_i$を定められるため、そちらでもいい??
この場合、
\begin{equation}
  \begin{array}{c}
    \dot{x}_1 = z_2\\
    \dot{z}_2 = v_2\\
    \{x_3 =  -\{ \frac{\partial f_1}{\partial x_1}(x_1)( f_1(x_1) + x_2)+f_2(x_1,x_2) \}+v_2 \}\\
    \vdots\\
    \dot{x}_{n-1} = z_n\\
    \dot{z}_n = v_n\\
    \begin{aligned}
      \{u = &-\{\frac{\partial f_{n-1}}{\partial x_1}(x_1,\dots,x_{n-1})(f_1(x_1)+x_2)
      + \frac{\partial f_{n-1}}{\partial x_2}(x_1,\dots,x_{n-1})(f_1(x_1,x_2)+x_3)
      \\
      & +\cdots +
      \frac{\partial f_{n-1}}{\partial x_{n-1}}(x_1,\dots,x_{n-1})(f_{n-1}(x_1,x_2,\dots,x_{n-1})+x_n)
      + f_n(x_1,x_2,\dots,x_n)\} + v\}\\
    \end{aligned}
  \end{array}
\end{equation}

\end{document}