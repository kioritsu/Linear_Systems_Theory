\documentclass{jsarticle}

\usepackage{amssymb}
\usepackage{graphicx}
\usepackage[dvipdfmx]{color}
\usepackage{here}
\usepackage{tabularx}
\usepackage{amsmath}
\usepackage{url}
\usepackage[hang,small,bf]{caption}
\usepackage[subrefformat=parens]{subcaption}
\usepackage{tikz}
\usepackage{siunitx}
\usepackage{bm}
\usepackage{ascmac}
\usepackage[top=15truemm,bottom=20truemm,left=20truemm,right=20truemm]{geometry}
\usetikzlibrary{shapes.geometric}
\usetikzlibrary {shapes.misc}
\usetikzlibrary{positioning}
\captionsetup{compatibility=false}


\begin{document}

\rightline{2022/6/14}
\section*{EX6.5:上三角行列の指数}
上三角行列$A$を考える。
\begin{itemize}
  \item[(a)] $e^{tA}$も上三角行列であることを示せ
  \item[(b)] $A$の対角要素と$e^{tA}$の対角要素を関連づけよ。
\end{itemize}
ヒント:行列の指数の定義を利用せよ。
\;\\
(a)行列指数の定義より、
\begin{equation}
  e^{tA} = \sum^{\infty}_{k=0} \frac{t^k}{k!}A^k \;,\;\forall t \in \mathbb{R} 
\end{equation}
上三角行列の累乗は上三角行列であり、上三角行列の和も上三角行列である。
$\frac{t^k}{k!}$は定数関数のため、$e^{tA}$も上三角行列である。\\
(b)\;\;
$A$の対角要素を$a_{jj}$、$e^{tA}$の対角要素を$e^{tA}_{jj}$とする。
上三角行列の累乗の対角要素は、${(A^k)}_{jj} = {(a_{jj})}^k$を満たすため、
\begin{equation}
  e^{tA}_{jj} = \sum^{\infty}_{k=0} \frac{t^k}{k!}{(a_{jj})}^k \;,\;\forall t \in \mathbb{R} 
\end{equation}

\;\\

上三角行列の性質の説明。\\
$n\times n$行列$A$が上三角行列であるとき、$a_{ij} = 0 (i>j)$である。\\
\;\\
上三角行列$A$と$B$の和の$i>j$である要素について考える。
\begin{equation}
  a_{ij} + b_{ij} = 0+0 = 0
\end{equation}
$i>j$である要素は0なので上三角行列$A$と$B$の和も上三角行列\\
\;\\
上三角行列$A$の2乗の$i>j$である要素について考える。
\begin{equation}
  {(A^2)}_{ij} = \sum^n_{k=1} a_{ik}a_{kj} = \sum^j_{k=1}  a_{ik}a_{kj} + \sum^n_{k=j+1}  a_{ik}a_{kj} = 0
\end{equation}
第一項については$a_{ik} = 0$第二項については$a_{kj}=0$、
$i>j$である要素は0なので上三角行列$A$の2乗も上三角行列\\
\;\\
上三角行列$A$の2乗の$i=j$の要素について考える。
\begin{equation}
  {(A^2)}_{jj} = \sum^n_{k=1} a_{jk}a_{kj} = \sum^{j-1}_{k=1}  a_{jk}a_{kj} +a_{jj}a_{jj}+ \sum^n_{k=j+1}  a_{jk}a_{kj} = {(a_{jj})}^2
\end{equation}
第一項については$a_{jk} = 0$第三項については$a_{kj}=0$、上三角行列の対角要素について、${(A^2)}_{jj}={(a_{jj})}^2$が言える。

\end{document}