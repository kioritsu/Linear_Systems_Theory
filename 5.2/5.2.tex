\documentclass{jsarticle}

\usepackage{amssymb}
\usepackage{graphicx}
\usepackage[dvipdfmx]{color}
\usepackage{here}
\usepackage{tabularx}
\usepackage{amsmath}
\usepackage{url}
\usepackage[hang,small,bf]{caption}
\usepackage[subrefformat=parens]{subcaption}
\usepackage{tikz}
\usepackage{siunitx}
\usepackage{bm}
\usepackage{ascmac}
\usepackage[top=15truemm,bottom=20truemm,left=20truemm,right=20truemm]{geometry}
\usetikzlibrary{shapes.geometric}
\usetikzlibrary {shapes.misc}
\usetikzlibrary{positioning}
\captionsetup{compatibility=false}


\begin{document}

\rightline{2022/6/1}

\section*{5.2 :非斉次線形システムの解}

非斉次LTV系である式(5.6)の解を求める。
\begin{equation}
  \dot{x} = A(t)x + B(t)u\;,\;y=C(t)x(t)+D(t)u\;,\;x(t_0)=x_0\in \mathbb{R}^n\;,\;t\geq 0 \tag{5.6}
\end{equation}

\begin{itembox}[l]{定理5.2:定数の変化}
式(5.6) の一意解は 
\begin{equation}
  x(t) = \Phi(t,t_0)x_0 + \int^t_{t_0} \Phi(t,\tau)B(\tau)u(\tau)d\tau \tag{5.7}
\end{equation}
\begin{equation}
  y(t) = C(t)\Phi(t,t_0)x_0 + \int^t_{t_0} C(t)\Phi(t,\tau)B(\tau)u(\tau) d\tau +D(t)u(t) \tag{5.8}
\end{equation}
で与えられる。ここで、$\Phi(t,t_0)$は状態遷移行列である。
\end{itembox}

式(5.7)は定数の変化式(variation of constants formula)として知られている。
式(5.8) の 
\begin{equation}
  y_h(t) := C(t)\Phi(t,t_0)x_0
\end{equation}
の項を斉次応答(homogeneous response)と呼び、
\begin{equation}
  y_f(t) := \int^t_{t_0} C(t)\Phi(t,\tau)B(\tau)u(\tau) d\tau +D(t)u(t)
\end{equation}
の項を強制応答(forced response)と呼ぶ。

定理 5.2 の証明\\
式(5.7) が 式(5.6) の解であることを確認するために、
$t = t_0$ において 式(5.7) の積分は消え、
$x (t_0) = x_0$ となることに注意する。
式(5.7) の各辺を時間に関して微分すると 
\begin{equation}
  \begin{aligned}
    \dot{x} = & \frac{d \Phi(t,t_0)}{dt} x_0 + \Phi(t,t)B(t)u(t) + \int^t_{t_0}\frac{d \Phi(t,\tau)}{dt}B(\tau)u(\tau)d\tau
    = & A(t)\Phi(t,t_0)x_0 + B(t)u(t) + A(t)\int^t_{t_0}\Phi(t,\tau)B(\tau)u(\tau) d\tau
    = & A(t)x(t) + B(t)u(t)
  \end{aligned}
\end{equation}
となり、式(5.7)は確かに式(5.6)の解であることが分かる。
解の一意性は、$x \mapsto A(t)x + B(t)u(t)$ が
任意の固定$t$に対してグローバル・リプシッツ写像であることから分かる。[1,Chapter1]

式(5.8) の $y (t)$ の式は、$y ( t ) = C ( t ) x ( t ) + D ( t ) u $に
$ x ( t )$ を代入することで得られる。 
$\Box$
\end{document}